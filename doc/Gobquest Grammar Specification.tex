\documentclass[11pt]{article}

\title{Gobquest grammar specification}
\author{Gobdev Team}
\date{\today}

\usepackage[utf8]{inputenc}
\usepackage{graphicx}
\usepackage[noend]{algpseudocode}
\usepackage{mathtools}
\usepackage{color}
\usepackage{caption}
\usepackage{fancyhdr}
\usepackage[english]{babel}
\usepackage{natbib}
\usepackage{listings}
\usepackage{amsthm}
\usepackage{amsmath}
\usepackage{amssymb}
\usepackage{titlesec}
\usepackage{hyperref}

\begin{document}

    \maketitle
    \newpage

    \tableofcontents
    \newpage

    \section{Flags}
        The flags in Gobquest are used to deny access to certain points
        before some requirements have been completed. Flags can be set
        as a result of a quest step completion or NPC interaction. They
        can be checked when selecting which exits and NPC's should be
        available in a room, and which conversation options should be
        available with an NPC. The default value given to a flag is
        the value the flag will have when the game is started with an
        empty save file.

        \subsection{Grammar}
            \begin{tabular}{l c l}
                start          & = & flag start \textbar\ EOF \\
                flag           & = & name default\_value \\
                name           & = & STRING \\
                default\_value & = & true \textbar\ false
            \end{tabular}
    \newpage

    \section{Rooms}
        Rooms in Gobquest contain all the other objects (with flags being
        an exception, since they are a program state). Each room will also
        have connections to other rooms, containing the name of the exit
        that is shown to the player, and the room name that this exit
        takes the player to. These exits can vary in a room
        depending on which flags are set for the player, which means that
        areas can be locked off until a certain point in the game is
        reached. The same thing can be applied to NPC's, locking certain
        NPC interactions until it is appropriate in the game. Note the
        tokens INDENT and DEDENT in the grammar; these tokens mean that
        the code is indented one more step and that one indentation is
        removed, respectively.
        \subsection{Grammar}
            \begin{tabular}{l c l}
                start  & = & TEXT ':\textbackslash n' INDENT name description
                             exits DEDENT start \textbar\ EOF \\
                name   & = & TEXT '\textbackslash n'
                             \textbar\ 'name:' nltext\\
                description & = & TEXT '\textbackslash n'
                             \textbar\ 'description:' nltext\\
                exits  & = & exit exits \textbar\ exit \\\\

                exit         & = & 'exit:' '\textbackslash n' INDENT exit\_name path requirements
                                   DEDENT \\
                exit\_name   & = & TEXT '\textbackslash n'
                                   \textbar\ 'name:' nltext \\
                path         & = & TEXT '\textbackslash n'
                                   \textbar\ 'path:' nltext \\
                requirements & = & flag '\textbackslash n' requirements
                                   \textbar\ $\epsilon$ \\\\
                nltext & = & TEXT '\textbackslash n' \textbar\
                             '\textbackslash n' INDENT TEXT '\textbackslash n'
                             DEDENT
            \end{tabular}



\end{document}
